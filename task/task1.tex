\documentclass[a4paper,12pt]{article}

% Pakete
\usepackage[utf8]{inputenc} 
\usepackage[T1]{fontenc} 
\usepackage{hyperref} % Für klickbare Links
\usepackage{titlesec} % Für ansprechende Überschriften
\usepackage{geometry} % Für Seitenränder
\usepackage{svg}      % Für SVG-Dateien
\usepackage{amsmath}
\geometry{a4paper, margin=2.5cm}

% Einstellungen für die Struktur
\titleformat{\section}{\bfseries\Large}{}{0em}{} 
\titleformat{\subsection}{\bfseries}{}{0em}{}

% Dokumentbeginn
\begin{document}

% Titelseite
\begin{titlepage}
    \vspace*{1cm}
    \bfseries\LARGE
    Werkzeuge für das wissenschaftliche Arbeiten\\[1em]
    \Large Python for Machine Learning and Data Science\\[2em]
    \normalsize\normalfont
    \hfill Abgabe: 15.12.2023
    
    \vfill
    \hrule
\end{titlepage}

% Inhaltsverzeichnis
\tableofcontents

% Abschnitt 1: Projektaufgabe
\section{Projektaufgabe}
In dieser Aufgabe beschäftigen wir uns mit Objektorientierung in Python. Der Fokus liegt auf der Implementierung einer Klasse, dabei nutzen wir insbesondere auch Magic Methods.

\begin{figure}[h!]
    \centering
    \includesvg[width=0.9\textwidth]{./../diagram/classes_files.svg}
    \caption{Darstellung der Klassenbeziehungen.}
    \label{fig:classes}
\end{figure}

\subsection{Einleitung}
Ein Datensatz besteht aus mehreren Daten. Ein einzelnes Datum wird durch ein Objekt der Klasse \texttt{DataSetItem} repräsentiert. Jedes Datum hat einen Namen (Zeichenkette), eine ID (Zahl) und beliebigen Inhalt.

Mehrere Daten, Objekte vom Typ \texttt{DataSetItem}, werden in einem Datensatz zusammengefasst. Es existiert eine Schnittstelle \texttt{DataSetInterface}, die die Operationen eines Datensatzes definiert. Ihre Aufgabe ist es, die Klasse \texttt{DataSet} als Unterklasse von \texttt{DataSetInterface} zu implementieren.

\subsection{Aufbau}
Die Implementierung erfolgt in drei Dateien*:
\begin{itemize}
    \item \texttt{dataset.py}: Enthält die Klassen \texttt{DataSetInterface} und \texttt{DataSetItem}.
    \item \texttt{implementation.py}: Hier erfolgt die Implementierung der Klasse \texttt{DataSet}.
    \item \texttt{main.py}: Testet die Klassen \texttt{DataSet} und \texttt{DataSetItem}.
\end{itemize}

\subsection{Methoden}
Folgende Methoden sind für die Klasse \texttt{DataSet} zu implementieren (Details finden Sie in \texttt{dataset.py}):

\begin{itemize}
    \item \texttt{\_\_setitem\_\_(self, name, id\_content)}: Hinzufügen eines Datums mit Name, ID und Inhalt.
    \item \texttt{\_\_iadd\_\_(self, item)}: Hinzufügen eines \texttt{DataSetItem}.
    \item \texttt{\_\_delitem\_\_(self, name)}: Löschen eines Datums anhand seines Namens (Namen sind eindeutige Schlüssel).
    \item \texttt{\_\_contains\_\_(self, name)}: Prüfung, ob ein Datum mit diesem Namen existiert.
    \item \texttt{\_\_getitem\_\_(self, name)}: Abrufen eines Datums über seinen Namen.
    \item \texttt{\_\_and\_\_(self, dataset)}: Schnittmenge zweier Datensätze berechnen und zurückgeben.
    \item \texttt{\_\_or\_\_(self, dataset)}: Vereinigungen zweier Datensätze berechnen und zurückgeben.
    \item \texttt{\_\_iter\_\_(self)}: Iteration über alle Daten im Datensatz (optional sortiert).
    \item \texttt{filtered\_iterate(self, filter)}: Gefilterte Iteration mithilfe einer Lambda-Funktion mit Parametern Name und ID.
    \item \texttt{\_\_len\_\_(self)}: Anzahl der Daten im Datensatz abrufen.
\end{itemize}

% Abschnitt 2: Abgabe
\section{Abgabe}
Die Implementierung der Klasse \texttt{DataSet} erfolgt in der Datei \texttt{implementation.py}. Nutzen Sie das Virtual Programming Lab (VPL) oder programmieren Sie lokal, indem Sie die Dateien aus Moodle herunterladen.

Im VPL wird \texttt{main.py} mit zusätzlichen Testfällen erweitert, um sicherzustellen, dass Ihre Implementierung korrekt ist.

\vfill
\hrule
\footnotesize{$^*$ Dateien befinden sich im Ordner \texttt{/code/} dieses Git-Repositories.}

\end{document}